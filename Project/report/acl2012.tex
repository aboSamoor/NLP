%
% File acl2012.tex
%
% Contact: Maggie Li (cswjli@comp.polyu.edu.hk), Michael White (mwhite@ling.osu.edu)
%%
%% Based on the style files for ACL2008 by Joakim Nivre and Noah Smith
%% and that of ACL2010 by Jing-Shin Chang and Philipp Koehn


\documentclass[11pt]{article}
\usepackage{acl2012}
\usepackage{times}
\usepackage{latexsym}
\usepackage{amsmath}
\usepackage{multirow}
\usepackage{url}
\DeclareMathOperator*{\argmax}{arg\,max}
\setlength\titlebox{6.5cm}    % Expanding the titlebox

\title{Detecting English Writing Styles For Non Native Speakers}

\author{Rami Al-Rfou', Yanging Chen, Yejin Choi \\
  Department of Computer Science \\
  Stony Brook University \\
  NY 11794, USA \\
  {\tt \{ralrfou, cyanqing, ychoi\}@cs.stonybrook.edu}}

\date{12/15/2011}

\begin{document}
\maketitle
\begin{abstract}
%Rami's Part
%Explain what is our target of whole project.
\end{abstract}


\section{Introduction}
%Rami's Part

\section{Related Work}
%Chen's part
%Mentions what people did, and how our work is different and original.
  The first work related with native language identification is that of Koppel et al. (2005), in which they tried profiling anonymous authors with their native languages. Totally five different groups of English authors (whose native languages are Russian, Bulgarian, French, and Spanish) were picked from the first version of {\em International Corpus of Learner English} (ICLE) in their experiments. By applying a combined feature sets, including function words, character n-grams, part-of-speech bi-grams and spelling mistakes, they gained an accuracy of 65\% if considered style features only. These results suggested that syntactic features are valuable when trying to categorize authors by their native languages.  
  Similar work was done by Tsur and Rappoport (2007). They focused on the relationships between choice of words in second language writing and the frequency of native language syllables, also known as the phonology of native languages. And Estival et al. (2007). studied a wide range of lexical and document structure features in their native laguages classification task. But either of these two mentioned the usefulness of syntacitc features for the task of native language detection.
   

\section{Wikipedia}
%Explain why wikipedia is an awesome source
%Explain wikipedia structure
%Explain how there are two different ways to get users contributions
%Explain the comments extraction algorithm
%Rami's Part

\section{Experiments}
%Rami's part

\subsection{Setup}
%Explain the pruning and the filtering that was done to the data.

\subsection{Popular Languages Experiment}
%Explain the experiment and the results


\subsection{Languages Families Experiment}
%Explain the experiment and the results


\subsection{Native vs Non Native Experiment}
%Explain the experiment and the results


\section{Writing Styles}
%Chen's part


\section{Conclusions}
%Chen's Part

\section*{Acknowledgments}
%Chen's Part
%We should thank Prof. Skiena for the resources he gave us.

\begin{thebibliography}{}

\bibitem[\protect\citename{Aho and Ullman}1972]{Aho:72}
Alfred~V. Aho and Jeffrey~D. Ullman.
\newblock 1972.
\newblock {\em The Theory of Parsing, Translation and Compiling}, volume~1.
\newblock Prentice-{Hall}, Englewood Cliffs, NJ.

\bibitem[\protect\citename{{American Psychological Association}}1983]{APA:83}
{American Psychological Association}.
\newblock 1983.
\newblock {\em Publications Manual}.
\newblock American Psychological Association, Washington, DC.

\bibitem[\protect\citename{{Association for Computing Machinery}}1983]{ACM:83}
{Association for Computing Machinery}.
\newblock 1983.
\newblock {\em Computing Reviews}, 24(11):503--512.

\bibitem[\protect\citename{Chandra \bgroup et al.\egroup }1981]{Chandra:81}
Ashok~K. Chandra, Dexter~C. Kozen, and Larry~J. Stockmeyer.
\newblock 1981.
\newblock Alternation.
\newblock {\em Journal of the Association for Computing Machinery},
  28(1):114--133.

\bibitem[\protect\citename{Gusfield}1997]{Gusfield:97}
Dan Gusfield.
\newblock 1997.
\newblock {\em Algorithms on Strings, Trees and Sequences}.
\newblock Cambridge University Press, Cambridge, UK.

\end{thebibliography}

\end{document}
